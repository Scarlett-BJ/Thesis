\clearpage
\pagestyle{empty}
\cleardoublepage
\pagestyle{fancy}

\chapter{Introduction}
\label{cha:introduction}

All life competes for the limited resources available. In this struggle, defense against pathogens is important for survival. Therefore, organisms from all kingdoms of life developed different defense mechanisms from the antibiotics of fungi against bacteria to the complex immune system of vertebrates. The more complex the organism is, the more diverse are the pathogens it has to face. Bacteria may be infected by viruses or other bacteria. However, human pathogens include viruses, bacteria, pathogenic fungi and other parasites, e.g., tapeworms. The pathogens attack the organism at different sites and in different ways. Therefore, the immune system needs to discriminate between self, which should be maintained, and non-self, which should be eliminated, i.e., induce an immune response. A substance triggering a specific immune response is called antigen.\\
The basic defense against a pathogen is the physical barrier of the epithelium. If the pathogen overcomes this barrier, e.g., through an injury, cells of the innate immune system try to eliminate it. In addition, these cells present protein fragments of the encountered pathogen on a specialized complex called major histocompatibility complex (MHC) on the cell surface. MHC-bound peptides are recognized by T~cells, which are part of the adaptive immune system, by means of their T~cell receptors (TCRs). Each T~cell carries only one sort of TCR with distinct specificity, but the complete T~cell population covers a broad range of potential antigens. However, T~cells directed against self peptides must not be present. Only the T~cells whose TCRs strongly interact with the peptide:MHC complex are activated. The activated T~cells replicate and initiate a highly specific immune response. The adaptive immune system is slower than the innate immune system in a first contact situation because the screening for cognate T~cells and their replication needs time. If the pathogen is encountered again, the adaptive immune system reacts faster and more efficiently. This is due to memory cells, generated during the replication of the cognate T~cells. The faster and more efficient reaction permitted by the immunological memory leads to a reduced impact of the pathogen on the organism. This immunological memory might even lead to immunity against the pathogen.\\
The generation of memory cells and their positive effects are used in vaccination. A vaccine includes one or more antigens that allow adaptation and building of memory cells without harming the organism. Such a vaccine may contain a weakened pathogen, a dead pathogen, or parts of a pathogen. After vaccination, a pathogen, which might otherwise not have been fought successfully, will be overwhelmed by the immune system. Therefore, vaccines offer a good protection against pathogens. Despite intensive attempts, it has not been possible to develop vaccines against some major pathogens including HIV. Peptide-based vaccines, containing specifically the immunogenic peptides, represent one line of research to gain new vaccines. To develop such a peptide-based vaccine, it is essential to know which of all possible peptides of the target pathogen are immunogenic. This knowledge is also important for non-peptide-based vaccine development. The regions of the pathogen containing immunogenic peptides should not be modified or deleted to weaken the pathogen during vaccine production.\\
Furthermore, peptide immunogenicity is of great interest to the research on autoimmune diseases. In these diseases, the adaptive immune system reacts against peptides derived from self-proteins, called auto-antigens. These auto-antigens are immunogenic, but should not elicit an immune response in a healthy organism. To examine this behavior, it is important to understand the properties of peptides that determine their immunogenicity. If immunogenicity is understood, this may lead to the development of advanced autoimmune treatments. The present autoimmune treatments are restricted to unspecific local or systemic immune suppression, which often has severe side effects.\\
