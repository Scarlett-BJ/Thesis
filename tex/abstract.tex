\pdfbookmark[0]{Abstract}{cha:abstract}
\chapter*{Abstract}
\thispagestyle{empty} 

In the struggle for survival, the immune system defends the body against pathogens. The cytotoxic T~cells of the adaptive immune system recognize peptides from within the cell that are presented on MHC class~I molecules on the cell surface. These peptides originate from self-proteins, but also from viral or bacterial proteins in the case of an infection. Thus, the cytotoxic T~cells monitor the intracellular space and discriminate between self and non-self. If non-self is detected within a cell, the cytotoxic T~cell kills the cell.\\
Although the immune system successfully fights most pathogens, some are able to overwhelm the immune response. Vaccines offer protection against these pathogens by allowing adaptation to the pathogen. Unfortunately, so far it has not been possible to develop vaccines against some major pathogens, such as HIV. Vaccine development would strongly benefit from knowing which of all possible peptides of the target pathogen are immunogenic. Furthermore, peptide immunogenicity is of great interest to the research on autoimmune diseases and may lead to the development of treatments superior to unspecific immune suppression. The amount of peptides whose immunogenicity would be of interest is too large for experimental determination with feasible time and effort. Therefore, computational predictions of immunogenicity are needed. Yet, no accurate immunogenicity predictor has been available, probably because immunogenic peptides are very diverse and have little conformity.\\
In this thesis, an SVM-based predictor for HLA-A*0201-restricted immunogenicity was developed. For training the predictor, a large set of immunogenicity and accessory data was collected from various sources and stored in the newly developed database \textit{pimDB}. In addition, biological knowledge on negative T~cell selection was incorporated in the predictor to improve its performance. \textit{In vivo} negative T~cell selection removes all self-reactive T~cells, and thus is one of the most important factors determining immunogenicity. The final predictor \textit{ANOPIM} classified 81\% of the immunogenic binders correctly, while identifying 89\% of the non-immunogenic ones. In total, it predicted the immunogenicity correctly for 86\% of the HLA-A*0201 bound peptides.
