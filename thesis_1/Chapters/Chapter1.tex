% Chapter 1

\chapter{Introduction} % Main chapter title
\label{Chapter1} % For referencing the chapter elsewhere, use \ref{Chapter1} 

%----------------------------------------------------------------------------------------
As the primary glucose transporter across the endothelial cells of the blood-brain barrier, the facilitated glucose transporter member 1 (GLUT1) protein plays a central role in the regulation of brain energy metabolism and maintenance of central nervous system homeostasis~\cite{Pascual}. Human GLUT1 is encoded by the \textit{SLC2A1} gene on the short arm of chromosome 1 at position 34.2, consist of 492 amino acids and contains 12 transmembrane domains~\cite{MUECKLER,Uldry}.


many of which have been reported to be susceptible to missense mutation that causes GLUT1 deficiency syndrome (G1DS), a genetic disease characterized by hypoglycorrhachia (low glucose concentration in the cerebrospinal fluid), seizures and delayed neurological development~\cite{Pascual,De}. GLUT1 mutations in G1DS patients impair glucose transport into the brain across the blood-brain barrier, resulting in the disease phenotypes. 

% treatment, discovery of this mutation, previous research

One of the clinically identified missense mutations in GLUT1 is a Pro485-to-Leu substitution (GLUT1\textsuperscript{P485L}) located in the cytoplasmic carboxyl tail~\cite{Pascual,Slaughter}. However, the molecular mechanisms by which the mutation causes the disease remains elusive. In a previous proteomic screen study to investigate the impact of disease-causing mutations, the GLUT1\textsuperscript{P485L} mutation was found to lead to an increased binding of clathrins. Sequence analysis revealed that the mutation creates a dileucine motif known to mediate clathrin-dependent endocytosis ([D/E]XXXL[L/I]) in the cytoplasmic tail~\cite{Pandey,Dinkel}.
%peptide-based interaction screen on disease-causing mutations

%TPEELFHLLGADSQV
% [D/E]XXXL[L/I] 

%A previous study of the group combined high-throughput peptide pull-downs and quantitative mass spectrometry~\cite{Meyer} to study binding profiles of proteins with disease related mutations. The results have shown that a peptide on the carboxyl terminal of GLUT1 with Pro485-to-Leu mutation (TPEELFHLLGADSQV) binds with clathrin, and this interaction is absent in the binding profile of the wild type peptide. Accordantly, the mutation creates a short linear motif (E...LL) in the cytoplasmic tail region of GLUT1 that interacts with adaptor proteins such as APs and mediates clathrin-dependent endocytic sorting~\cite{Traub}. Furthermore, a proximity-dependent biotin identification (BioID) experiment~\cite{Roux} in HEK293 cells transiently transfected with GLUT1 provides additional evidence for the novel interaction between the mutant GLUT1 and several proteins involved in the clathrin-mediated endocytosis pathway.

Based on these findings, it is hypothesized that the GLUT1\textsuperscript{P485L} mutation causes clathrin-mediated endocytosis and possibly subsequent degradation of GLUT1, leading to the development of GLUT1-deficiency syndrome. The hypothesis will be further investigated in this master thesis.

% Define some commands to keep the formatting separated from the content 
\newcommand{\keyword}[1]{\textbf{#1}}
\newcommand{\tabhead}[1]{\textbf{#1}}
\newcommand{\code}[1]{\texttt{#1}}
\newcommand{\file}[1]{\texttt{\bfseries#1}}
\newcommand{\option}[1]{\texttt{\itshape#1}}


