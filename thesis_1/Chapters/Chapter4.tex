% Chapter 4

\chapter{Discussion} % Main chapter title
\label{Chapter4} % For referencing the chapter elsewhere, use \ref{Chapter4} 
\addtocontents{toc}{\setcounter{tocdepth}{1}}
GLUT1 serves as the primary glucose transporter across vascular endothelial cells of the human blood-brain barrier~\cite{Pascual}. Mutations in GLUT1 can cause impaired glucose transport into the brain, leading to G1DS~\cite{Klepper.2}. A new GLUT1 mutation, GLUT1\textsuperscript{P485L}, was reported in 2009 in a child with G1DS, but its pathogenic mechanisms remain unknown~\cite{Slaughter}.

In a recent study by our group, a novel interaction was discovered between a short peptide in the carboxyl terminal of GLUT1 carrying the P485L mutation and clathrin, but this interaction is absent in the binding profile of the corresponding wild-type peptide~\cite{Meyer2}. Furthermore, sequence analysis revealed that the mutation creates a novel short linear motif ([DE]XXXL[LI]) that is known to mediate endocytosis and intracellular trafficking of membrane proteins~\cite{Meyer2,Bonifacino}. In HeLa cells transiently transfected with GLUT1, the wild-type GLUT1 was localized to the plasma membrane, whereas the mutant GLUT1 showed vesicular localization and colocalization with endocytosed transferrin~\cite{Meyer2}. Moreover, BioID experiments in HEK293 cells with GLUT1 transient expression showed increased colocalization of the mutant GLUT1 with proteins associated with clathrin-mediated endocytosis and endosomal trafficking~\cite{Meyer2}. These results suggested that the GLUT1\textsuperscript{P485L} mutation causes internalization of the protein via clathrin-mediated endocytosis~\cite{Meyer2}. 

In this thesis, we used two HEK293 stable cell lines with stable inducible GLUT1 expression to further investigate the effect of the GLUT1\textsuperscript{P485L} mutation on the intracellular localization and trafficking of the protein.

\section{Subcellular localization of GLUT1}
In agreement with previous observations in HeLa cells transiently transfected with GLUT1, confocal images of GLUT1 localization in HEK293 cells stably expressing FLAG-GLUT1 confirmed the mislocalization of the mutant GLUT1 in intracellular vesicles~\cite{Meyer2}. Furthermore, the mutant GLUT1 showed colocalization with early endosomal markers (EEA1, Rab5), a late endosomal marker (Rab9), and markers for the trans-Golgi network (Vti1a, Vti1b).
% Rab7, LAMP1
% quantification
% bioid
%\section{Intracellular trafficking of GLUT1}
% motif
\section{Leakiness and loss of GLUT1 expression}
%When culturing cells in medium containing fetal bovine serum (FBS), please note that many lots of FBS contain tetracycline as FBS is generally isolated from cows that have been fed a diet containing tetracycline. If you culture your cells in medium containing FBS that is not reduced in tetracycline, you may observe low basal expression of your gene of interest in the absence of tetracycline

%Tetracycline (MW = 444.4) is commonly used as a broad spectrum antibiotic and acts to inhibit translation by blocking polypeptide chain elongation in bacteria. In the T-REx? System, tetracycline is used as an inducing agent to induce transcription of the gene of interest from the inducible expression vector. Tetracycline induces transcription by binding to the Tet repressor homodimer and causing the repressor to undergo a conformational change that renders it unable to bind to the Tet operator. The association constant of tetracycline to the Tet repressor is 3 x 109 M-1 (Takahashi et al., 1991). 

%Tetracycline is light sensitive

%You may want to vary the concentration of tetracycline (0.1 to 1 �g/ml) and time of exposure to tetracycline (8 to 24 hours) to optimize or modulate expression for your cell line.

%Stable cell lines often lose their protein expression with time as a result of a heterogeneity in the transfected population of cells. A more homogeneous population of cells can be obtained by limiting dilution cloning or picking individual colonies of drug-resistant cells.

% selection used by the landhaler lab; one population might overgrow another, causing the increased leaky expression over time.

Because the loss of expression over time was observed in a sub-population of both wild-type and mutant cells (Figure%figure! 
, new cells were thaw and the expression of GLUT1 variants were analyzed again with Western blotting and immunofluorescence microscopy. (Supplementary Figure%figure!

\section{Conclusion and outlook}
%----------------------------------------------------------------------------------------
% Define some commands to keep the formatting separated from the content 
