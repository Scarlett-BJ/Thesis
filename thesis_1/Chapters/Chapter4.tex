% Chapter 4

\chapter{Discussion} % Main chapter title
\label{Chapter4} % For referencing the chapter elsewhere, use \ref{Chapter4} 
\addtocontents{toc}{\setcounter{tocdepth}{1}}
GLUT1 serves as the primary glucose transporter across vascular endothelial cells of the human blood-brain barrier~\cite{Pascual}. Mutations in GLUT1 can cause impaired glucose transport into the brain, leading to G1DS~\cite{Klepper.2}. A new GLUT1 mutation, GLUT1\textsuperscript{P485L}, was reported in 2009 in a child with G1DS, but its pathogenic mechanisms remain unknown~\cite{Slaughter}.

In a recent study by our group, a novel interaction was discovered between a short peptide in the carboxyl terminal of GLUT1 carrying the P485L mutation and clathrin, but this interaction is absent in the binding profile of the corresponding wild-type peptide~\cite{Meyer2}. Furthermore, sequence analysis revealed that the mutation creates a novel short linear motif ([DE]XXXL[LI]) that is known to mediate endocytosis and intracellular trafficking of membrane proteins~\cite{Meyer2,Bonifacino}. In HeLa cells transiently transfected with GLUT1, the wild-type GLUT1 was localized to the plasma membrane, whereas the mutant GLUT1 showed vesicular localization and colocalization with endocytosed transferrin~\cite{Meyer2}. Moreover, BioID experiments in HEK293 cells with GLUT1 transient expression showed increased colocalization of the mutant GLUT1 with proteins associated with clathrin-mediated endocytosis and endosomal trafficking~\cite{Meyer2}. These results suggested that the GLUT1\textsuperscript{P485L} mutation causes internalization of the protein via clathrin-mediated endocytosis~\cite{Meyer2}. 

In this thesis, we used two HEK293 stable cell lines with stable inducible GLUT1 expression to further investigate the effect of the GLUT1\textsuperscript{P485L} mutation on the intracellular localization and trafficking of the protein.

\section{Subcellular localization of the mutant GLUT1}
In agreement with previous observations in HeLa cells transiently transfected with GLUT1, confocal images of GLUT1 localization in HEK293 cells stably expressing FLAG-GLUT1 confirmed the mislocalization of the mutant GLUT1 in intracellular vesicles~\cite{Meyer2}. In contrast, the wild-type GLUT1 showed a predominantly plasma membrane distribution. This differential localization of the GLUT1 variants has been confirmed by immunostaining using an antibody against GLUT1 (Figure~\ref{fig:glut1}).

Furthermore, the subcellular distribution of the mutant GLUT1 was compared with the distribution of several markers for intracellular compartments. We found that the mutant GLUT1 partially localized to early endosomes, as labeled by classical early markers EEA1 and Rab5 (Figure~\ref{fig:ee}), and with late endosomes, based on Rab9 staining. However, the mutant GLUT1 was significantly less colocalized with another late endosomal marker Rab7 in comparison to Rab9 (Figure~\ref{fig:le}). This is because Rab7 and Rab9 are localized to distinct domains on late endosomes and have different functional roles in intracellular trafficking~\cite{Barbero}. Rab7 regulates the selective sorting of cargos from early endosomes to late endosomes and transport from late endosomes to lysosomes, whereas Rab9 regulates transport from late endosomes to the trans-Golgi network~\cite{Girard,Barbero}. 

Because Rab9 has been reported to be removed upon membrane fusion of late endosomes and the Golgi complex, additional labeling with markers for the trans-Golgi network was performed and confirmed the Golgi localization of the mutant GLUT1 (Figure~\ref{fig:tgn})~\cite{Barbero}. While both being SNARE proteins and homologous to the yeast Q-SNARE Vti1p, the two markers used in this thesis, Vti1a and Vti1b, have overlappping but distinct subcellular distributions~\cite{Kreykenbohm}. Vti1a has been shown to participate in the receipt of cargos arriving at the trans-Golgi network from early endosomes and late endosomes, the latter of which requires Rab9 and the recruitment of its effects~\cite{Ganley}. Vti1b has been reported to show a broader distribution among different organelles, including early endosomes, late endosomes, the trans-Golgi network and lysosomes, and its roles include participation in late endosome fusion, late endosome-lysosome fusion and vesicle budding from the trans-Golgi network~\cite{Kreykenbohm,Murray}. The involvement of Vti1a and Vti1b in different intracellular trafficking pathways makes it difficult to interpret the possible trafficking routes of the mutant GLUT1 from the colocalization studies. However, the colocalization of the mutant GLUT1 with Rab9 as well as Vti1a indicates that the mutant GLUT1 might be a cargo of the retrograde transport from late endosomes to the trans-Golgi network.

The lack of colocalization between the mutant GLUT1 and recycling endosomal markers suggests that neither the fast nor the slow recycling pathway plays a crucial role in the trafficking of the mutant GLUT1 (Figure~\ref{fig:re}). It is therefore interesting to check whether the mutant GLUT1 undergoes lysosomal degradation. Co-immunostaining did not show much colocalization of the mutant GLUT1 with LAMP1 (Figure~\ref{fig:le}), which is a transmembrane protein mainly localized on lysosomes~\cite{Schroder}. However, the quantification of colocalization suggests that there might be a weak correlation between GLUT1 and LAMP1 distribution ($r=0.11$, Figure~\ref{fig:coloc})). Besides, the inhibition of lysosomal degradation with Bafilomycin A1 also seems to increase the mutant GLUT1 protein levels as assessed by Western blotting (Figure~\ref{fig:wb2}). Replicate confocal images for colocalization analysis and replicate experiments of Bafilomycin inhibition are needed to better characterize whether the mutant GLUT1 is partially shuttled to lysosomes for degradation. Additionally, it would be interesting to see whether the mutant GLUT1 would colocalize more extensively with LAMP1 after Bafilomycin A1 treatment to block lysosomal degradation.

Consistent with the colocalization analysis of confocal fluorescence microscopy images, BioID experiments showed that the mutant GLUT1 specifically colocalizes with proteins involved in the retrograde transport (SNX1, SNX2, SNX3, SNX6, AP1B1, CLINT1, SMAP2 and IGF2R) (Figure~\ref{fig:bioid2}). The sorting nexins (SNX1, SNX2 etc) mediate the fusion of incoming endosomes with the trans-Golgi network, whereas AP1, CLINT1 and SMAP2 regulate the formation of clathrin-coated vesicles (CCV) and their subsequent budding from the trans-Golgi network~\cite{Seaman,Bonifacino,Funaki,Legendre}.  No protein specifically associated with endosomal recycling or lysosomal degradation was identified in the experiments. An identified protein LRBA has been suggested to localize in recycling endosomes and the trans-Golgi network in T cells, but its function is still unclear~\cite{Lo}. Rab1 and Rab6 mediate ER-Golgi trafficking and intra-Golgi trafficking, respectively, and their possible involvement in the trafficking of the mutant GLUT1 remains to be determined~\cite{Stenmark}. Although these proteins were specifically identified in both the forward and reverse experiments, data from biological replicate BioID experiments might help to further reduce noise and define the significance level of the identified colocalization.
% However, a proximal protein to the mutant GLUT1 identified in the experiments, GOPC, has been reported to interact with \textbeta1-adrenergic receptors in the trans-Golgi network and protect them from lysosomal degradation~\cite{Koliwer}. Another protein, SCAMP1, has been suggested to facilitate endocytosis by recruiting clathrin coats to the plasma membrane and the trans-Golgi network~\cite{doi: 10.1074/jbc.275.17.12752, 2000}. 

The results from the BioID experiments and confocal immunofluorescence microscopy consistently indicated that the GLUT1\textsuperscript{P485L} mutation causes clathrin-mediated endocytosis of the mutant GLUT1 and its mislocalization to the trans-Golgi network via endosomal pathways, including retrograde transport. Although the proportion of the mutant GLUT1 distribution in intracellular compartments remains to be quantified, its extensive colocalization with transferrin/EEA1/Rab9 suggests that the mutant GLUT1 is not restricted to a stable storage compartment. The colocalization of the mutant GLUT1 with regulators of CCV formation such as AP1 and CLINT1 suggests that the mutant GLUT1 might be sorted at the trans-Golgi network to other compartments such as lysosomes or back to the plasma membrane.
%Because the wild-type GLUT1 did not show perinuclear localization, it is unlikely that the retention of the mutant GLUT1 at the trans-Golgi network is
% motif comparison
% post-Golgi? Rab11
% Dileucine-based motifs are recognized by AP-1, AP-2, and AP-3; however, each AP complex exhibits distinct preferences for certain [DE]XXXL[LI] motifs.

The canonical dileucine motif in the cytosolic tails of proteins targeted to endosomal and lysosomal compartments, [DE]XXXL[LI], is recognized by AP1, AP2 and AP3, which modulate CCV formation and mediate cargo protein sorting~\cite{McNiven}. For instance, the dileucine motif in the CD3\textgamma chain is known to participate in the serine phosphorylation-dependent down-regulation of T cell receptors, a process involving rapid clathrin-dependent endocytosis and lysosomal degradation of the receptors~\cite{Dietrich}. GLUT8 has been shown to be retained in a late endosomal/lysosomal compartment by sorting signals of its N-terminal EXXXLL motif~\cite{Diril}. The dileucine sequence (LL) proximal to the carboxyl terminal of the Menkes protein (MNK) has been shown to provide the endocytic signal required for the internalization of MNK from the plasma membrane and its targeting to the trans-Golgi network~\cite{Petris,Petris2}. The endosomal and Golgi localization of the mutant GLUT1 characterized in this thesis is in agreement with previous studies about the function of the dileucine motif. 
%studies have shown that an acidic residue is not critical in this location for adaptor protein interactions~\cite{Bonifacino}.
% bioid vs KM's bioid
% wT trafficking

%\section{Leakiness and loss of GLUT1 expression}
%When culturing cells in medium containing fetal bovine serum (FBS), please note that many lots of FBS contain tetracycline as FBS is generally isolated from cows that have been fed a diet containing tetracycline. If you culture your cells in medium containing FBS that is not reduced in tetracycline, you may observe low basal expression of your gene of interest in the absence of tetracycline

%Tetracycline (MW = 444.4) is commonly used as a broad spectrum antibiotic and acts to inhibit translation by blocking polypeptide chain elongation in bacteria. In the T-REx? System, tetracycline is used as an inducing agent to induce transcription of the gene of interest from the inducible expression vector. Tetracycline induces transcription by binding to the Tet repressor homodimer and causing the repressor to undergo a conformational change that renders it unable to bind to the Tet operator. The association constant of tetracycline to the Tet repressor is 3 x 109 M-1 (Takahashi et al., 1991). 

%Tetracycline is light sensitive

%You may want to vary the concentration of tetracycline (0.1 to 1 �g/ml) and time of exposure to tetracycline (8 to 24 hours) to optimize or modulate expression for your cell line.

%Stable cell lines often lose their protein expression with time as a result of a heterogeneity in the transfected population of cells. A more homogeneous population of cells can be obtained by limiting dilution cloning or picking individual colonies of drug-resistant cells.

% selection used by the landhaler lab; one population might overgrow another, causing the increased leaky expression over time.

%Because the loss of expression over time was observed in a sub-population of both wild-type and mutant cells (Figure%figure! , new cells were thaw and the expression of GLUT1 variants were analyzed again with Western blotting and immunofluorescence microscopy. (Supplementary Figure%figure!

\section{Conclusion and outlook}
GLUT1\textsuperscript{P485L} is a missense mutation in the glucose transporter GLUT1 characterized in a patient with G1DS~\cite{Slaughter}. This mutation results in the appearance of a new dileucine motif at the cytosolic carboxyl tail of GLUT1, which is known to mediate clathrin-dependent trafficking of membrane proteins~\cite{Meyer2,Bonifacino}. 

In this thesis we studied the subcellular localization and trafficking of the wild-type and the mutant GLUT1 in HEK293 stable inducible cells. While the wild-type GLUT1 is predominantly localized to the plasma membrane, the mutant GLUT1 mislocalizes to intracellular compartments and colocalizes with endocytosed clathrin. SILAC-based quantitative characterization of proximate proteins identified increased colocalization of the mutant GLUT1 with proteins involved in clathrin-mediated endocytosis and proteins associated with intracellular compartments, especially the trans-Golgi network. Consistently, confocal fluorescence microscopy analysis confirmed the colocalization of the mutant GLUT1 with early endosomes, late endosomes and the trans-Golgi network. Taken together, our data suggest that the GLUT1\textsuperscript{P485L} mutation causes internalization of the GLUT1 protein via clathrin-mediated endocytosis. The mislocalization of GLUT1 could lead to impaired glucose transport into the brain and thus the development of G1DS.

%Live cell imaging
The subcellular localization analyses in this thesis were performed with HEK293 cells stably expressing BirA- and FLAG-tagged wild-type or mutant GLUT1. To exclude possible influence caused by overexpression of GLUT1 and epitope tagging on the subcellular localization of GLUT1 and disturbance of intracellular compartments, the colocalization analysis could be repeated on the endogenous mutant GLUT1 in patient-derived cells, especially patient-derived brain microvascular endothelial cells as they would provide a more clinically relevant model system. The difficulties in accessibility of brain-derived endothelial cells could be overcome with the assistance of the induced pluripotent stem cell (iPSC) technology~\cite{Minami}. 
% by generating induced pluripotent stem cells (iPSCs) from patient-derived somatic tissues and inducing their differentiation 

The results presented in this thesis suggest that the dileucine motif created by the GLUT1\textsuperscript{P485L} mutation at the carboxyl tail recruits AP2 and thus leads to the internalization of the protein. More studies are required to characterize the functional involvement of the dileucine motif and AP2 in the GLUT1 relocation. For instance, the effect of the novel dileucine motif in the mutant GLUT1 could be determined with mutational analysis of the conserved amino acid residues in the motif. The signals contained in this motif would be abrogated by substituting either of the critical leucines by alanine~\cite{Bonifacino}. Additionally, the critical role of AP2 could be examined in AP2 knockdown studies with siRNA oligos.

Finally, functional analysis of the mutant GLUT1 could provide additional information about the pathogenicity of the GLUT1\textsuperscript{P485L} mutation. The degree of glucose transport reduction could be determined by measuring the glucose uptake kinetics of the wild-type and the mutant GLUT1, and might explain the clinical phenotypes of the G1DS patient with the GLUT1\textsuperscript{P485L} mutation.
%It was notable that mutation of the L1487L1488 signal did not fully abolish perinuclear labelling of MNK. This was probably due to retention of newly synthesized MNK in the TGN via the third transmembrane domain, recently identified as containing a putative TGN retention signal 
%----------------------------------------------------------------------------------------
% Define some commands to keep the formatting separated from the content 
