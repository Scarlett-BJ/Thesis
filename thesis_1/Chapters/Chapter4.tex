% Chapter 4

\chapter{Discussion} % Main chapter title
\label{Chapter4} % For referencing the chapter elsewhere, use \ref{Chapter4} 

%When culturing cells in medium containing fetal bovine serum (FBS), please note that many lots of FBS contain tetracycline as FBS is generally isolated from cows that have been fed a diet containing tetracycline. If you culture your cells in medium containing FBS that is not reduced in tetracycline, you may observe low basal expression of your gene of interest in the absence of tetracycline

%Tetracycline (MW = 444.4) is commonly used as a broad spectrum antibiotic and acts to inhibit translation by blocking polypeptide chain elongation in bacteria. In the T-REx? System, tetracycline is used as an inducing agent to induce transcription of the gene of interest from the inducible expression vector. Tetracycline induces transcription by binding to the Tet repressor homodimer and causing the repressor to undergo a conformational change that renders it unable to bind to the Tet operator. The association constant of tetracycline to the Tet repressor is 3 x 109 M-1 (Takahashi et al., 1991). 

%Tetracycline is light sensitive

%You may want to vary the concentration of tetracycline (0.1 to 1 �g/ml) and time of exposure to tetracycline (8 to 24 hours) to optimize or modulate expression for your cell line.

%Doxycycline has been shown to have a longer half-life than tetracycline (48 hours vs. 24 hours, respectively). 
%----------------------------------------------------------------------------------------
% Define some commands to keep the formatting separated from the content 
