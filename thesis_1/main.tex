%%%%%%%%%%%%%%%%%%%%%%%%%%%%%%%%%%%%%%%%%
% Masters/Doctoral Thesis 
% LaTeX Template
% Version 2.4 (22/11/16)
%
% This template has been downloaded from:
% http://www.LaTeXTemplates.com
%
% Version 2.x major modifications by:
% Vel (vel@latextemplates.com)
%
% This template is based on a template by:
% Steve Gunn (http://users.ecs.soton.ac.uk/srg/softwaretools/document/templates/)
% Sunil Patel (http://www.sunilpatel.co.uk/thesis-template/)
%
% Template license:
% CC BY-NC-SA 3.0 (http://creativecommons.org/licenses/by-nc-sa/3.0/)
%
%%%%%%%%%%%%%%%%%%%%%%%%%%%%%%%%%%%%%%%%%
%----------------------------------------------------------------------------------------
%	PACKAGES AND OTHER DOCUMENT CONFIGURATIONS
%----------------------------------------------------------------------------------------

\documentclass[
12pt, % The default document font size, options: 10pt, 11pt, 12pt
%oneside, % Two side (alternating margins) for binding by default, uncomment to switch to one side
english, % ngerman for German
onehalfspacing, % Single line spacing, alternatives: onehalfspacing or doublespacing
%draft, % Uncomment to enable draft mode (no pictures, no links, overfull hboxes indicated)
%nolistspacing, % If the document is onehalfspacing or doublespacing, uncomment this to set spacing in lists to single
%liststotoc, % Uncomment to add the list of figures/tables/etc to the table of contents
%toctotoc, % Uncomment to add the main table of contents to the table of contents
%parskip, % Uncomment to add space between paragraphs
%nohyperref, % Uncomment to not load the hyperref package
headsepline, % Uncomment to get a line under the header
%chapterinoneline, % Uncomment to place the chapter title next to the number on one line
%consistentlayout, % Uncomment to change the layout of the declaration, abstract and acknowledgements pages to match the default layout
%longbibliography,
]{MastersDoctoralThesis} % The class file specifying the document structure

\usepackage[utf8]{inputenc} % Required for inputting international characters
\usepackage[T1]{fontenc} % Output font encoding for international characters
\usepackage{siunitx}
\usepackage{times} % Use the Palatino font by default
\usepackage{textgreek}
\usepackage{textcomp}
%\usepackage{titlesec}
\usepackage{caption}
\usepackage[calcwidth = \linewidth, labelfont = bf, textfont=bf]{caption}
\usepackage{ragged2e}
\usepackage[section]{placeins}
\usepackage{commath}
\usepackage{subcaption}
%\usepackage{physics}
%\usepackage[backend=bibtex,style=authoryear,natbib=true]{biblatex} % Use the bibtex backend with the authoryear citation style (which resembles APA)

%\addbibresource{main.bib} % The filename of the bibliography

%\usepackage[autostyle=true]{csquotes} % Required to generate language-dependent quotes in the bibliography

\usepackage[sort&compress,numbers]{natbib}
\usepackage{verbatim}
\bibliographystyle{apsrev4-1}
%\usepackage{doi}%<----------
%\usepackage{hyperref}
\def\bibsection{\section*{\refname}} 
\sisetup{detect-all}
\captionsetup{font=normalsize,width=\columnwidth}
%\titleformat*{\section}{\Large\bfseries}
%----------------------------------------------------------------------------------------
%	MARGIN SETTINGS
%----------------------------------------------------------------------------------------

\geometry{
	paper=a4paper, % Change to letterpaper for US letter
	inner=2.5cm, % Inner margin
	outer=2.5cm, % Outer margin
	bindingoffset=.5cm, % Binding offset
	top=2.5cm, % Top margin
	bottom=2.5cm, % Bottom margin
	%showframe, % Uncomment to show how the type block is set on the page
}

%----------------------------------------------------------------------------------------
%	THESIS INFORMATION
%----------------------------------------------------------------------------------------

\thesistitle{Characterizing the Role of a GLUT1 Mutation in GLUT1-deficiency Syndrome} % Your thesis title, this is used in the title and abstract, print it elsewhere with \ttitle
\supervisor{Prof. Dr. Matthias \textsc{Selbach}} % Your supervisor's name, this is used in the title page, print it elsewhere with \supname
\examiner{} % Your examiner's name, this is not currently used anywhere in the template, print it elsewhere with \examname
\degree{Master of Science} % Your degree name, this is used in the title page and abstract, print it elsewhere with \degreename
\author{Jingyuan \textsc{Cheng}} % Your name, this is used in the title page and abstract, print it elsewhere with \authorname
\addresses{} % Your address, this is not currently used anywhere in the template, print it elsewhere with \addressname

\subject{Molecular Medicine} % Your subject area, this is not currently used anywhere in the template, print it elsewhere with \subjectname
\keywords{} % Keywords for your thesis, this is not currently used anywhere in the template, print it elsewhere with \keywordnames
\university{Charit\'{e} Universit\"{a}tsmedizin Berlin} % Your university's name and URL, this is used in the title page and abstract, print it elsewhere with \univname
\department{Protein Dynamics} % Your department's name and URL, this is used in the title page and abstract, print it elsewhere with \deptname
\group{Protein Dynamics} % Your research group's name and URL, this is used in the title page, print it elsewhere with \groupname
\faculty{} % Your faculty's name and URL, this is used in the title page and abstract, print it elsewhere with \facname

\AtBeginDocument{
\hypersetup{pdftitle=\ttitle} % Set the PDF's title to your title
\hypersetup{pdfauthor=\authorname} % Set the PDF's author to your name
\hypersetup{pdfkeywords=\keywordnames} % Set the PDF's keywords to your keywords
}

\begin{document}

\frontmatter % Use roman page numbering style (i, ii, iii, iv...) for the pre-content pages

\pagestyle{plain} % Default to the plain heading style until the thesis style is called for the body content

%----------------------------------------------------------------------------------------
%	TITLE PAGE
%----------------------------------------------------------------------------------------

\begin{titlepage}
\begin{center}

\begin{figure}
\centering
\includegraphics[scale=1.0]{Figures/Charite.pdf}\ \\[2ex]

\Large\bfseries Master's Program in Molecular Medicine\\
\Large at the Charit\'{e} - Universit\"{a}tsmedizin Berlin\\
\vspace*{1cm}
\includegraphics[scale=1.0]{Figures/molmed.pdf}
\label{titlepictures}
\end{figure}

\LARGE\bfseries Master's Thesis\\
\large\normalfont to earn the\\
\LARGE\bfseries Master of Science in Molecular Medicine\\

\vspace*{1cm}

\LARGE\bfseries Characterizing the Role of a GLUT1 Mutation in GLUT1-deficiency Syndrome

\vspace*{1cm}

\large\normalfont Presented by\\
\bfseries Jingyuan Cheng\\
\large\normalfont Born on May 17th, 1993\\

\vspace*{1cm}
\end{center}
\begin{flushleft}
\large\normalfont First Evaluator: Professor Dr. Matthias Selbach\\
Second Evaluator: Professor Dr. Volker Haucke\\

\vspace*{1cm}

Completed at AG Selbach, Max Delbr\"{u}ck Center for Molecular Medicine
\end{flushleft}

\begin{comment}

\vspace*{.06\textheight}
{\scshape\LARGE \univname\par}\vspace{1.5cm} % University name
% \textsc{\Large Doctoral Thesis}\\[0.5cm] % Thesis type

\HRule \\[0.4cm] % Horizontal line
{\huge \bfseries \ttitle\par}\vspace{0.4cm} % Thesis title
\HRule \\[1.5cm] % Horizontal line
 
\begin{minipage}[t]{0.4\textwidth}
\begin{flushleft} \large
\emph{Author:}\\
\authorname % Author name - remove the \href bracket to remove the link
\end{flushleft}
\end{minipage}
\begin{minipage}[t]{0.4\textwidth}
\begin{flushright} \large
\emph{Supervisor:} \supname % Supervisor name - remove the \href bracket to remove the link  
\end{flushright}
\end{minipage}\\[3cm]
 
\vfill

\large \textit{A thesis submitted in fulfillment of the requirements\\ for the degree of \degreename}\\[0.3cm] % University requirement text
\textit{in the}\\[0.4cm]
%\groupname\\\deptname\\[2cm] % Research group name and department name
 
\vfill

{\large \today}\\[4cm] % Date
%\includegraphics{Logo} % University/department logo - uncomment to place it

\end{comment}
 
\vfill
\end{titlepage}

%----------------------------------------------------------------------------------------
%	ABSTRACT PAGE
%----------------------------------------------------------------------------------------

\begin{abstract}
\addchaptertocentry{\abstractname} % Add the abstract to the table of contents
%need modification
Glucose transporter-1 (GLUT1) deficiency syndrome is a genetic disorder characterized by impaired glucose transport into the brain. One of the clinically identified pathogenic mutations is a Pro485-to-Leu substitution located in the cytoplasmic carboxyl tail of GLUT1, whose pathogenic mechanisms remain unclear. A previous \textit{in vitro} proteomic screen from our group revealed that this GLUT1 mutation leads to specific interactions with clathrins, which is supported by the further bioinformatic finding that the mutation creates a novel dileucine motif known to mediate clathrin-dependent trafficking. \\
\\*
In this study we used stable inducible HEK293 cells to further investigate the effect of the GLUT1 mutation on the intracellular localization and trafficking of the protein. We showed that the wild-type GLUT1 mainly localizes to the plasma membrane, whereas the mutant GLUT1 mislocalizes to intracellular compartments and co-localizes with endocytosed transferrin, as well as early endosomal and late endosomal markers. Moreover, SILAC-based quantitative characterization of proximate proteins also identified increased proximate interaction of the mutant GLUT1 with proteins associated with endocytosis and endosomal compartments. Together, these data suggest that the GLUT1\textsuperscript{P485L} mutation causes internalization of the GLUT1 protein via clathrin-mediated endocytosis, thus leading to GLUT1 deficiency syndrome.\\

%On the other hand, the mutant GLUT1 shows no co-localization with the lysosomal marker LAMP1, and the inhibition of lysosomal degradation cannot increase the mutant GLUT1 level. These results suggest that the mutant GLUT1 

\end{abstract}

%----------------------------------------------------------------------------------------
%	LIST OF CONTENTS/FIGURES/TABLES PAGES
%----------------------------------------------------------------------------------------

\tableofcontents % Prints the main table of contents

\listoffigures % Prints the list of figures

\listoftables % Prints the list of tables

%----------------------------------------------------------------------------------------
%	THESIS CONTENT - CHAPTERS
%----------------------------------------------------------------------------------------

\mainmatter % Begin numeric (1,2,3...) page numbering

\pagestyle{thesis} % Return the page headers back to the "thesis" style

% Include the chapters of the thesis as separate files from the Chapters folder
% Uncomment the lines as you write the chapters

% Chapter 1

\chapter{Introduction} % Main chapter title
\label{Chapter1} % For referencing the chapter elsewhere, use \ref{Chapter1} 
\addtocontents{toc}{\setcounter{tocdepth}{1}}

%----------------------------------------------------------------------------------------
As the primary glucose transporter across the endothelial cells of the blood-brain barrier, the facilitated glucose transporter member 1 (GLUT1) protein plays a central role in the regulation of brain energy metabolism and maintenance of central nervous system homeostasis~\cite{Pascual}. Human GLUT1 is encoded by the \textit{SLC2A1} gene on the short arm of chromosome 1, consists of 492 amino acids and contains 12 transmembrane \textalpha{} helices (Figure~\ref{fig:topo})~\cite{MUECKLER,Uldry}. GLUT1 is highly expressed in endothelial cells and glial cells, but is ubiquitously expressed at lower levels as well~\cite{Lee,Wheeler}. Two isoforms of GLUT1 have been found, namely the 55kDa form with N-linked glycosylation at Asn45 and the 45kDa unglycosylated form~\cite{Paul-W.-Hruz,Duelli}.
\begin{figure}[h]
\centering
\includegraphics[scale=0.7]{Figures/topology}
\caption{GLUT1 topology on the plasma membrane.}
\vspace*{-3mm}
\small \justify
Identified sites of missense mutations are indicated with numbered circles, including the site of the P485L mutation. The figure is adapted from \textit{Structural signatures and membrane helix 4 in GLUT1}, by J. M. Pascual and D. Wang et al, 2008, the Journal of Biological Chemistry~\cite{Pascual}.
\label{fig:topo}
\end{figure}

% gene: at position 34.2
% glycosylation outside the brain??????
Mutations in the \textit{SLC2A1} gene can result in GLUT1 deficiency syndrome (G1DS), an autosomal dominant disorder caused by impaired GLUT1-mediated glucose transport into the brain~\cite{De,Klepper}. To date, approximately 80 mutations in the \textit{SLC2A1} gene have been detected in about 140 patients, including large-scale deletions, insertions, missense, nonsense, frame shift, translation initiation and splice-site mutations~\cite{Wang, Leen}. These mutations are heterozygous and result in GLUT1 haploinsufficiency - absence or loss of a functional allele~\cite{Klepper,Leen}. The classic G1DS phenotype consists of intractable epilepsy presenting in infancy, delayed neurologic development, secondary microcephaly and complex movement disorders~\cite{De,Klepper}. Milder variants have been reported to affect about 10\% G1DS patients and present mental retardation, movement abnormalities but without clinical seizures~\cite{Wang,Suls}. The diagnostic hallmark of G1DS features reduced cerebrospinal fluid (CSF) glucose concentration (hypoglycorrhachia) combined with low CSF lactate concentration and low 3-O-methyl-D-glucose uptake in erythrocytes~\cite{Wang,Klepper.2}. The ketonic diet, introduced as a treatment for G1DS in 1991, provides an alternative fuel for brain metabolism and effectively controls the seizures in G1DS patients~\cite{Wang}.

 \section{The GLUT1\textsuperscript{P485L} mutation}

% discovery of this mutation, previous research
A \textit{de novo} pathogenic mutation in GLUT1, the Pro485Leu (P485L) mutation, was reported in 2009 in a child with G1DS~\cite{Slaughter}. This is a point mutation in exon 10 of the gene, leading to the missense mutation in the cytoplasmic carboxyl tail of the protein (Figure~\ref{fig:topo}). Low CSF glucose concentration was detected in a lumbar puncture, and the patient presented intractable infantile-onset epilepsy and mild developmental delay~\cite{Slaughter}. However, little is known about the molecular mechanisms by which this mutation causes G1DS.

A previous study in our group combined high-throughput peptide pull-downs and quantitative mass spectrometry to investigate the impact of 128 disease-causing mutations in intrinsically disordered regions on protein-protein interaction~\cite{Meyer2}. Cellulose membranes with synthetic wild-type and mutant peptides (15 amino acids in length) were incubated with cell lysates to pull down interacting proteins. In the subsequent mass spectrometry analysis, 3 mutations in disordered cytosolic regions of 3 transmembrane proteins, including the GLUT1\textsuperscript{P485L} mutation, were identified to specifically interact with clathrin, the major coat protein involved in clathrin-mediated endocytosis~\cite{Meyer2}. Moreover, sequence analysis revealed that all three mutations were proline to leucine changes and resulted in the appearance of a novel dileucine motif (Figure~\ref{fig:motif})~\cite{Meyer2}. Two of the three mutations created the classic dileucine motif, [Asp-Glu]X-X-X-Leu[Leu-Ile] ([DE]XXXL[LI]), which is known to mediate the rapid internalization of transmembrane proteins and their targeting to late endosomes, lysosomes and lysosome-related organelles~\cite{Bonifacino}. Variations of the amino acid residue at position -4 from the first leucine in this motif have been reported in several transmembrane proteins, and this residue might be important for targeting to different intracellular compartments~\cite{Bonifacino,Sandoval}.
%The signaling effects of the motif can be abrogated by substituting either of the critical leucines to alanine
\begin{figure}[h]
\centering
\includegraphics[scale=0.7]{Figures/motif}
\caption{Gain of dileucine motifs in the mutated peptides.}
\label{fig:motif}
\end{figure}

Based on these findings, it was hypothesized that the GLUT1\textsuperscript{P485L} mutation causes clathrin-mediated endocytosis and possibly the subsequent lysosomal degradation of GLUT1, leading to the development of G1DS. This hypothesis and the intracellular trafficking of the mutant GLUT1 was further investigated in this thesis.

\section{Clathrin-mediated endocytosis and endocytic pathways}
Clathrin-mediated endocytosis is a process by which eukaryotic cells internalize material from the cell surface through clathrin-coated vesicles~\cite{McMahon}. Its fundamental roles include regulating turnover of plasma membrane proteins and lipids, taking up nutrients into the cell, regulating signaling pathways, etc~\cite{McMahon,Humphries}. Apart from clathrin-mediated endocytosis, there are also other pathways that mediate endocytosis, such as caveolin-dependent internalization, and clathrin- and caveolin-independent internalization.
\begin{figure}[h]
\centering
\includegraphics[scale=0.7]{Figures/endocytosis}
\caption{Clathrin-mediated endocytosis and intracellular trafficking.}
\vspace*{-3mm}
\small \justify
A. A simplified illustration of the clathrin-dependent internalization. The cargo is not shown in the figure. The figure is adapted from \textit{Molecular mechanism and physiological functions of clathrin-mediated endocytosis}, by H. T. McMahon and E. Boucrot, 2011, Nature Reviews Molecular Cell Biology~\cite{McMahon}. B. The intracellular compartments and trafficking. The figure is adpated from \textit{Endosome maturation}, by J. Huotari and A. Helenius, 2011, the EMBO Journal~\cite{Huotari}. 
\label{fig:endocytosis}
\end{figure}

Clathrin-mediated endocytosis consists of five stages: initiation, cargo selection, coat assembly, scission and uncoating. Clathrin does not directly bind to the cargo (the transmembrane receptor is taken as an example in the following) and thus requires the recruitment of adaptor proteins (APs), such as AP2, to the site where the clathrin-coated vesicle is about to form. AP2 specifically acts at the plasma membrane, whereas the other adaptor proteins, AP1, AP3 and AP4, are involved in the formation and budding of clathrin-coated vesicles on endosomes and the trans-Golgi network~\cite{Hirst2,McMahon}. APs fulfills the adaptor function through binding both with clathrin and the cargo receptor. The dileucine motif created in the cytoplasmic tail of the mutant GLUT1, [DE]XXXL[LI], is one of the most common motifs that AP1, AP2 and AP3 can recognize and bind to~\cite{Humphries}. Other sorting motifs recognized by APs include Asn-Pro-X-Tyr and Tyr-X-X-hydrophobic residue~\cite{McNiven}. Further accessory proteins are recruited to the site and promote membrane invagination and vesicle formation. The budding of the clathrin-coated vesicle requires scission by the GTPase dynamin~\cite{McMahon}. Once detached from the parent membrane, the vesicle moves away from the membrane and the clathrin coat is disassembled. The uncoated vesicle then travels to and fuses with the early endosome (Figure~\ref{fig:endocytosis} A)~\cite{McMahon}.

The early endosome is the main sorting station in the endocytic pathway~\cite{Huotari}. It receives vesicles from clathrin-dependent and independent pathways, and the cargo can be then recycled back to the cell surface, to the trans-Golgi network, or routed along the degradative pathway to late endosomes and lysosomes (Figure~\ref{fig:endocytosis} B)~\cite{Huotari}. The small GTPase Rab5 is a key component of the early endosome. It has been shown to play an essential role in the assembly of clathrin-coated vesicles at the plasma membrane, the heterotypic fusion of endocytic vesicles with endosomes, the homotypic fusion of early endosomes, and the conversion of early endosomes to late endosomes~\cite{Stenmark,Huotari}. 

Endocytic uptake and endocytic recycling are essential to the regulation of the plasma membrane composition. Two recycling routes are responsible for the transport of cargos back to the cell surface: the fast recycling route from early endosomes or an earlier stage to the plasma membrane, and the slow recycling route from early endosomes to the plasma membrane via recycling endosomes~\cite{Grant}. Rab4 has been identified as an important regulator of the fast recycling route, but its precise function remains unclear~\cite{Grant}. Rab11 plays a key role in the slow recycling route~\cite{Grant}. The transferrin receptor (TRFC) is the most intensively studied cargo protein of clathrin-mediated endocytosis and slow recycling. 

The degradative pathway following endocytosis involves the transport of cargo proteins from early endosomes to late endosomes and then to endolysosomes, which are generated through the fusion of late endosomes and lysosomes. Rab7 is recruited to early endosomes by Rab5 and replaces Rab5, marking the formation of new late endosomes~\cite{Huotari}. Rab7 mediates maturation of late endosomes, the homotypic fusion of late endosomes, and the heterotypic fusion of late endosomes with lysosomes~\cite{Stenmark}. The maturation of late endosomes is accompanied by their acidification by regulating the concentration and isoforms of membrane-bound V-ATPases~\cite{Huotari}. The luminal pH of endocytic organelles drops from 6.8-6.1 in early endosomes to 6.0-4.8 in late endosomes, whereas the pH in lysosomes is around 4.5~\cite{Maxfield}. As they mature, late endosomes move along microtubules to bring the cargo to the vicinity of lysosomes~\cite{Huotari}.

Cargo proteins can also be transported from early endosomes, late endosomes or recycling endosomes to the trans-Golgi network via retrograde transport~\cite{Huotari,Johannes}. The traffic between endosomes and the trans-Golgi network is bidirectional and continuously ongoing~\cite{Huotari}. The best studied examples of retrograde cargo proteins are mannose 6-phosphate receptors (MPRs) which transport newly synthesized lysosomal enzymes from the trans-Golgi network to late endosomes. The receptors dissociate with their ligands, the lysosomal enzymes, in the acidic endosomal compartment and are recycled back to the trans-Golgi network~\cite{Progida}. The transport is clathrin-dependent and requires additional regulators such as AP1 and GGAs, both of which can recognize sorting motifs on the cargo protein~\cite{Klinger}. The transport from early endosomes to the trans-Golgi network is mediated by the retromer complex composed of a cargo selective trimer (Vps35, Vps26 and Vps29) and a dimer of sorting nexins (SNX1 or SNX2 with SNX5 or SNX6)~\cite{Johannes,Seaman}. Rab9 mediates the retrieval of MPRs from late endosomes back to the trans-Golgi network~\cite{Stenmark}. Effector proteins are recruited by Rab9 to late endosomes and bind to the cytosolic tail of MPRs, thereby facilitating the sorting of MPRs into late endosomal recycling buds~\cite{Stenmark}. 

Previous research has shown that while localizing at the plasma membrane at steady state, the wild-type GLUT1 undergoes continuous clathrin-independent endocytosis and recycling from endosomes to the plasma membrane~\cite{Eyster,McGough}. It has been suggested that a small proportion of GLUT1 might traffic to early endosomes en route to late endosomes and lysosomes after being endocytosed~\cite{Eyster,McGough}. The majority of endocytosed GLUT1 is efficiently transported back to the cell surface as a SNX27-retromer cargo~\cite{Steinberg}.
% Define some commands to keep the formatting separated from the content 
\newcommand{\keyword}[1]{\textbf{#1}}
\newcommand{\tabhead}[1]{\textbf{#1}}
\newcommand{\code}[1]{\texttt{#1}}
\newcommand{\file}[1]{\texttt{\bfseries#1}}
\newcommand{\option}[1]{\texttt{\itshape#1}}

% Chapter 2

\chapter{Materials and methods} % Main chapter title
\label{Chapter2} % For referencing the chapter elsewhere, use \ref{Chapter2}

\bfseries{Cell line generation}\\
\normalfont Doxycycline-inducible wild-type and mutant GLUT1-expressing cell lines had been generated previously in the lab of Markus Landthaler at Max Delbr\"{u}ck Center. In brief, the gene \textit{SLC2A1} was purchased from Harvard Plasmid Repository and a stop codon was added with the primers listed in Table~\ref{tab:primers} (BioTeZ). The P485L mutation was introduced by PCR-directed mutagenesis with the primers listed in Table~\ref{tab:primers}. The wild-type and mutant \textit{SLC2A1} coding sequences were recombined into the vector pDEST-pcDNA5-BirA-FLAG using Gateway cloning system (ThermoFisher) (Figure~\ref{fig:vectors})~\cite{Couzens}. HEK293 Flp-In T-Rex cells (Invitrogen) were cotransfected with pOG44 Flp-recombinase expression vector (ThermoFisher) and the recombinant vector containing wild-type or mutant GLUT1 to generate stable cell lines. Two days following transfection, cells were selected with 100 \textmu g/mL Hygromycin B (InvivoGen) for 2 weeks. A control HEK293 cell line for BioID experiments had been generated in the lab using the identical technique with an integrated transgene for the inducible expression of the first 10 strands of GFP (GFP1-10).

\begin{table}[h]
%\captionsetup{font=normalsize}
\caption{Primer sequences for mutagenesis of GLUT1.}
\label{tab:primers}
\small
\centering
\begin{tabular*}{\textwidth}{l@{\extracolsep{\fill}}p{11.1cm}}
\toprule
\tabhead{Purpose} & \tabhead{Primer sequences (5' to 3')}\\
\midrule
Adding a stop codon & Forward: TCCCAAGTGTAATTGCCAACTTTCTTGTACAAAGTTG \newline Reverse: ATCAGCCCCCAGGGGATG\\
P485L mutation & Forward: CTGTTCCATCTCCTGGGGGCT \newline Reverse: CTCCTCGGGTGTCTTGTCAC\\
\bottomrule\\
\end{tabular*}
\end{table}
\begin{figure}[h]
%\captionsetup{font=normalsize}
\centering
\includegraphics[scale=0.9]{Figures/vectors}
\caption{Graphic map of the recombinant plasmids.}
\label{fig:vectors}
\end{figure}
The stable cell lines were stored in cryogenic vials (Croning) in liquid nitrogen. To recover cells, one vial of each cell line was warmed up to \SI{37}{\celsius} and the contents were diluted in 10 mL pre-warmed DMEM (Life Technologies) complemented with 10\% fetal bovine serum (Pan-Biotech). The medium is referred to as complete DMEM in the following. After spinning down at room temperature and 500 rpm for 3 min, the cell pellets were resuspended in 10 mL complete DMEM and seeded in T-75 flasks (CELLSTAR).
\\
\\*
\bfseries{Cell culture}\\
\normalfont Stable HEK293 cells were cultured at \SI{37}{\celsius} and 5\% CO\textsubscript{2} in culture flasks in complete DMEM. Cells were seeded to about 10\% confluency and routinely passaged twice a week as follows: the medium was aspirated and the cells were briefly washed with 4 mL sterile pre-warmed PBS (Life Technologies). To detach the cells 1 mL trypsin-EDTA (0.05\%, Life Technologies) was added and the flask was placed in an incubator at \SI{37}{\celsius} and 5\% CO\textsubscript{2} for 1 min. Trypsin was then inactivated with 9 mL pre-warmed complete DMEM. The medium was gently pipetted to the bottom of the flask in order to recover all the cells and get single-cell suspension. 1 mL of the cell suspension was transferred into 10 mL fresh complete DMEM in a new culture flask which was then placed back in the incubator. The cells would reach approximately 80\% confluency before the next passaging.

\begin{figure}[h]
\centering
\includegraphics[scale=0.7]{Figures/BioID}
\caption{Experimental setup of triple SILAC labels.}
\label{fig:bioid}
\end{figure}
Cells used for BioID experiments were cultured in SILAC DMEM (Life Technologies) complemented with 2 mM glutamine (Glutamax, Life Technologies), 1 mM pyruvate (Life Technologies), 0.1 mM non-essential amino acids (Life Technologies) and 10\% dialyzed fetal bovine serum (Pan-Biotech). In addition, L-arginine (Arg0, Sigma-Aldrich) and L-lysine (Lys0, Sigma-Aldrich) were added to the Light SILAC DMEM to a final concentration of 0.199 mM and 0.339 mM, respectively. Alternatively, Arg6 and Lys4 or Arg10 and Lys8 were added in place of their Light counterparts to make Medium-heavy SILAC DMEM and Heavy SILAC DMEM, respectively. PBS-EDTA (Lonza) was used to detach the cells when passaging to avoid isotope contamination. After six passages cells were fully labeled as assessed by MS.
% From Koshi: stock conc. are Lys 0.798M, Arg 0.398M (my dilution is 1:2000). DMEM conc: Lys 146mg/L, Arg 84mg/L; if 1:3000 dilution, final conc. should be 1/4 of standard DMEM conc.

For the doxycycline induction experiments unlabeled HEK293 cells were cultured in 6-well plates. The cells were grown to approximately 50\% confluency on the second day after seeding. The medium was removed and 2 mL complete DMEM containing doxycycline (Sigma-Aldrich) was carefully added to each well. After 24 hr or 48 hr the cells were harvested for Western blotting analysis.

For the BioID experiments fully labeled HEK293 cells were seeded in 15 cm plates with approximately 25\% density. Two plates were used for each condition. HEK293 stable cells expressing GFP1-10 were labeled in Light SILAC DMEM and used as negative control cells. Besides the Light control cells, in the forward experiment mutant GLUT1 cells were cultured in Medium-heavy SILAC DMEM and wild-type GLUT1 cells were cultured in Heavy SILAC DMEM (Figure~\ref{fig:bioid} A), while in the reverse label-swap experiment wild-type GLUT1 cells were cultured in Medium-heavy SILAC DMEM and mutant GLUT1 cells were cultured in Heavy SILAC DMEM (Figure~\ref{fig:bioid} B). The cells were grown to 40\%-50\% confluency before being treated with 0.1 {}\textmu g/mL doxycycline and 1 mM biotin (ThermoFisher). After 24 hr, the cells were scraped in ice-cold PBS and collected for further BioID purification and MS analysis.

For the immunofluorescence experiments sterile glass coverslips (Roth, 18 mm diameter, 0.170 mm thickness) were placed in 12-well plates. Poly-L-lysine (0.01\%, Sigma-Aldrich) was added to each well to cover the coverslips. After incubation at room temperature for 5 min, poly-L-lysine was recovered from the wells and stored at \SI{4}{\celsius}. The coverslips were washed twice with sterile H\textsubscript{2}O before being air-dried completely. Unlabeled HEK293 cells were then seeded in the plates with approximately 25\% density. The cells were treated with 0.1 {}\textmu g/mL doxycycline on the second day and subjected to subsequent immunostaining on the third day.
\\
\\*
\bfseries{Western blotting}\\
\normalfont Cells were grown in 6-well plates as described above. After doxycycline or inhibitor treatment, cells were scraped in ice-cold PBS and spun down at 300 rcf, \SI{4}{\celsius} for 4 min. Cell pellets were lyzed for 15 min at room temperature in 100 \textmu L lysis buffer [50 mM ABC solution, 2\% SDS, supplemented with 60 Units/mL benzonase (Sigma-Aldrich) and protease inhibitors (Roche)]. Lysates were spun down at 16 100 rcf for 15 min to remove cell debris and supernatants were transferred to new Eppendorf tubes. For each SDS-PAGE sample, 15 {}\textmu L supernatant was diluted in LDS sample buffer (NOVEX) complemented with 1 {}\textmu L 1M DTT (Sigma-Aldrich) before being heated at \SI{70}{\celsius} for 10 min in a thermoblock. Samples were then loaded onto a 4\%-12\% gel (NOVEX) along with protein ladders (PageRuler Plus, ThermoFisher). Proteins were separated using electrophoresis for 90 min at 150 V in 400 mL MES running buffer (ThermoFisher). 

Before protein transfer a PVDF membrane (Merck Millipore) was activated in methanol for 1 min and equilibrated in ice-cold transfer buffer (25 mM Tris-HCl, 192 mM glycine, 20\% methanol, pH 8.3) for 10 min. Whatman filter papers and sponges were also soaked in transfer buffer for 10 min. A tank blotting system (Invitrogen) was used to transfer the separated proteins to the membrane. In short, a transfer sandwich was prepared with the membrane on the cathode and the gel on the anode. Air bubbles between the gel and membrane were removed by rolling them out with a roller. The cassette was then placed in the transfer tank on ice and the proteins were transferred at a constant current of 250 mA for 2 hr.

After protein transfer the membrane was blocked in 5\% milk powder in TBST (150 mM sodium chloride, 20 mM Tris-HCl, 0.1\% Tween-20, pH 7.6) at room temperature for 30 min. The membrane was then incubated with the primary antibody diluted in blocking buffer while rotating at \SI{4}{\celsius}. The membrane was washed 3 times for 5 min in TBST before being incubated at room temperature for 1 hr with the HRP-conjugated secondary antibody diluted in blocking buffer. The membrane was washed again before the chemiluminescence substrate (PerkinElmer) was applied to the membrane. The chemiluminescent signals were captured using a ChemiDoc MP Imaging System (Bio-Rad) and quantified with Image Lab 5.2.1. The primary and secondary antibodies used in this thesis are summarized in Table~\ref{tab:antibodies}.
\begin{table}[h]
\caption{Antibodies used for Western blotting and their dilutions.}
\label{tab:antibodies}
\small
\centering
\begin{tabular*}{\textwidth}{l@{\extracolsep{\fill}}lll}
\toprule
\tabhead{Antibodies} & \tabhead{Source} & \tabhead{Dilution} & \tabhead{Conjugate}\\
\midrule
Rabbit polyclonal anti-FLAG & Cell Signaling Technology & 1:1000 & HRP\\
Rabbit polyclonal anti-LC3 & Novus Biologicals & 1:1000 & \\
Rat monoclonal anti-HA & Roche & 1:1000 & \\
Mouse monoclonal anti-\textbeta-actin & Sigma-Aldrich & 1:10 000 & \\
Donkey anti-rabbit IgG & GE Healthcare & 1:10 000 & HRP\\
Goat anti-rat IgG & GE Healthcare & 1:15 000 & HRP\\
Sheep anti-mouse IgG & GE Healthcare & 1:25 000 & HRP\\
\bottomrule\\
\end{tabular*}
\end{table}
\\*
\bfseries{Transfection and inhibitor treatments}\\
\normalfont Inhibitor treatment experiments were performed as the doxycycline induction experiments but only with the 24 hr induction time point. In addition to inducing GLUT1 expression with 0.1 \textmu g/mL doxycycline, different inhibitors or DMSO (Biomol) control were added. Lysosomal degradation was inhibited using 250 nM bafilomycin A1 (Invitrogen) for 6 hr before the cells were lysed for Western blotting analysis.

Cells for proteosomal inhibition experiments were transfected with UL21a-HA, an HA-epitope tagged HCMV gene, 1 hr after adding doxycycline. It has been reported that the short-lived viral protein pUL21a encoded by UL21a is targeted for proteosome-dependent degradation and can be drastically stabilized in the presence of proteasome inhibitor MG132 in fibroblasts~\cite{Fehr}. In brief, 20 \textmu g UL21a-HA was mixed with 40 \textmu L jetPRIME (Polyplus) in 2 mL jetPRIME (Polyplus) buffer. After incubation for 10 min, 200 \textmu L of the resulting transfection mix was added drop wise onto the cells in each well. On the next day proteasomes were blocked using 20 mM MG132 (Cayman chemical) for 6 hr. The cells were then lysed for Western blotting analysis.
\\
\\*
\bfseries{BioID and MS analysis}\\
\normalfont Cells scraped from 15 cm plates were spun down at 300 rcf at \SI{4}{\celsius} for 4 min. After resuspension in ice-cold PBS, the triple SILAC labels were combined into a forward and a reverse experiment (Figure~\ref{fig:bioid}). The cells were spun down again and the cell pellets were resuspended in 1.5 mL of RIPA buffer (50 mM tris-HCl, 150 mM NaCl, 1\% NP-40, 1 mM EDTA, 1 mM EGTA, 0.1\% SDS, 1\% sodium deoxycholate, pH 7.5, supplemented with 75 Units/mL benzonase and protease inhibitors). After incubation with agitation at \SI{4}{\celsius} for 1 hr, the lysates were sonicated on ice for 3 min. The lysates were then centrifuged for 20 min at 16 100 rcf at \SI{4}{\celsius} to remove cell debris and the supernatants were transferred to 2 mL Eppendorf tubes.

For affinity purification a 180 {}\textmu L bed volume of streptavidin T1 magnetic beads (Invitrogen) was washed in PBS twice and RIPA buffer once before being resuspended in 200 {}\textmu L RIPA buffer. 100 {}\textmu L of the bead solution was added to each lysate sample. Affinity purification was performed at \SI{4}{\celsius} for 3 hr on a rotating wheel.

The beads were then carefully washed twice in RIPA buffer and twice in TAP lysis buffer (50 mM HEPES-KOH, 100 mM KCl, 10\% glycerol, 2 mM EDTA, 0.1\% NP-40, pH 8.0). The detergents were removed by washing three times in 50 mM ABC. The beads were resuspended in 200 {}\textmu L ABC and 10 {}\textmu L of 10 mM DTT in 50 mM ABC was added to reduce disulfide bonds of proteins. After 30 min incubation in a thermomixer at \SI{30}{\celsius} at 500 rpm, 10 {}\textmu L of 55 mM IAA (Sigma-Aldrich) in 50 mM ABC was added to alkylate the cysteine residues and the samples were incubated in the dark at \SI{30}{\celsius} at 500 rpm for 30 min. The samples then were digested with 1.5 {}\textmu g trypsin (Promega) and incubated overnight at \SI{30}{\celsius} at 1100 rpm. On the next day, the tryptic peptides were separated from beads with a magnetic rack and transferred to fresh 1.5 mL tubes. The digestion was stopped by acidification with 10 \textmu L of 10\% TFA and the tryptic peptides were loaded on C18 StageTip columns for purification~\cite{Rappsilber}. After washing with sample buffer (3\% TFA, 5\% acetonitrile) twice, the peptides were eluted from the columns into an autosampler plate with 50 {}\textmu L buffer B (0,1\% formic acid, 80\% acetonitrile). A vacuum centrifuge (Eppendorf) was used to evaporate the solvent and 10 {}\textmu L buffer A (5\% acetonitrile, 0,1\% formic acid) was added to the samples.

Peptides were separated on a reverse-phase column on a High Performance Liquid Chromatography (HPLC) system (ThermoScientific) with a gradient set up as the following ratios of buffer B in buffer A: 2 min at 250 {}\textmu L/min with a linear gradient from 5\% to 6\% buffer B, 18 min at 200 {}\textmu L/min from 6\% to 8\%, 80 min at 200 {}\textmu L/min from 8\% to 20\%, 80 min at 200 {}\textmu L/min from 20\% to 33\%, 20 min at 200 {}\textmu L/min from 33\% to 45\%, 2 min at 200 {}\textmu L/min from 45\% to 60\%, 1 min at 250 {}\textmu L/min from 60\% to 95\%, 5 min at 250 {}\textmu L/min with 95\% buffer B, 1 min at 250 {}\textmu L/min from 95\% to 75\%, and 5 min at 250 {}\textmu L/min at 75\% buffer B. Peptides were ionized using an electrospray ionization source (ThermoScientific) and analyzed on a Q-Exactive Plus Orbitrap instrument (ThermoScientific). The mass spectrometer was operated with Xcalibur in data-dependent acquisition mode with the following parameters: a full MS scan (resolution: 70 000, scan range: 300 to 1700 m/z, AGC target: 1 000 000 ions, maximum injection time: 120 ms) followed by MS-MS analysis (resolution: 17 500, AGC target: 100 000 ions, maximum injection time: 60 ms) on the top 10 abundant ions.

The resulting raw files were analyzed using MaxQuant 1.5.2.8~\cite{Cox}. The reference human proteome database, consisting of 159 616 entries, was downloaded from the UniProt Knowledgebase on February 28th, 2017. Lys4, Arg6 and Lys8, Arg10 were added to the labels. Variable modifications were set as N-terminal acetylation and oxidation of methionines. The \textit{in silico} digestion of peptides in the reference database was performed with trypsin/P site-specific cleavage. Default global parameters were kept except that "re-quantify" and "match between runs" were turned on. The false discovery rate, assessed by in-parallel searching in a decoy database generated by reversing the reference database, was set to 0.01 at peptide and protein levels. The following analysis and graphics were performed using Perseus v1.5.8.4 and R v3.3.3.
% cut off
\\
\\*
\bfseries{Transferrin uptake}\\
\normalfont After 24 hr of doxycycline induction, the cells were starved for 1 hr in starvation buffer [serum-free medium supplemented with 20 mM HEPES (Life Technologies)]. Transferrin conjugated with Alexa Fluor 568 (Life Technologies) was diluted in starvation buffer to a final concentration of 10 {}\textmu g/mL and droplets of 35 {}\textmu L transferrin solution were pipetted onto a parafilm in a dark humid chamber. The coverslips were then carefully lifted off the bottom of the plate with forceps and placed face-down on the droplets. The humid chamber was incubated at \SI{37}{\celsius} and 5\% CO\textsubscript{2} for 10 min. After transferrin uptake, the coverslips were washed 3 times for 5 min with PBS supplemented with 10 mM MgCl\textsubscript{2} and 10 mM CaCl\textsubscript{2}, followed by standard immunostaining procedures described as below.
\\
\\*
\bfseries{Immunostaining and confocal microscopy}\\
\normalfont Prior to staining, cell culture medium was aspirated and cells were briefly washed with pre-warmed PBS. The cells were fixed in 4\% PFA for 15 min at room temperature before being washed 3 times for 5 min in PBS. To permeabilize cells and block unspecific binding sites the cells were incubated for 1 hr at room temperature in blocking buffer [5\% goat serum (Sigma-Aldrich), 0.3\% Triton X-100 (Sigma-Aldrich) in PBS]. The primary antibody was diluted in antibody dilution buffer [1\% BSA - Fraction V (Sigma-Aldrich), 0.3\% Triton X-100 in PBS], as indicated in Table~\ref{tab:IF}. The coverslips were placed face-down on droplets of 50 {}\textmu L primary antibody solution on a piece of parafilm in a dark humid chamber. After 1 hr incubation at room temperature, the coverslips were placed face-up in the plate and washed 3 times for 5 min in PBS. Similarly, the coverslips were incubated with secondary antibody solution for 1 hr at room temperature in the humid chamber before being washed for 5 min in PBS. The coverslips were then counterstained with DAPI (0.1 {}\textmu g/mL in PBS, Sigma-Aldrich) for 3 min and washed for 10 min in PBS. Finally, the coverslips were briefly rinsed in Milli-Q H\textsubscript{2}O and mounted with ProLong Gold Antifade Mountant (Life Technologies) on slides. After overnight incubation, the slides were stored at \SI{4}{\celsius} in the dark.

\begin{table}[h]
%\captionsetup{font=normalsize}
\caption{Antibodies for immunofluorescence and their dilutions.}
\label{tab:IF}
\small
\centering
\begin{tabular*}{\textwidth}{l@{\extracolsep{\fill}}lll}
\toprule
\tabhead{Antibodies} & \tabhead{Source} & \tabhead{Dilution} & \tabhead{Conjugate}\\
\midrule
Mouse monoclonal anti-FLAG & Sigma-Aldrich & 1:200 & \\
Rabbit polyclonal anti-GLUT1 & Merck Millipore & 1:500 & \\
Rabbit monoclonal anti-EEA1 & Cell Signaling Technology & 1:100 & \\
Rabbit monoclonal anti-Rab4 & Cell Signaling Technology & 1:100 & \\
Rabbit monoclonal anti-Rab9 & Cell Signaling Technology & 1:100 & \\
Rabbit monoclonal anti-Rab11 & Cell Signaling Technology & 1:100 & \\
Rabbit monoclonal anti-LAMP1 & Cell Signaling Technology & 1:100 & \\
Mouse monoclonal anti-Vti1a & BD Biosciences & 1:100\\
Mouse monoclonal anti-Vti1b & BD Biosciences & 1:100\\
Mouse monoclonal anti-STX6 & BD Biosciences & 1:100\\
Goat anti-mouse IgG (H+L) & Invitrogen & 1:500 & Alexa Fluor 488\\
Goat anti-rabbit IgG (H+L) & Invitrogen & 1:500 & Alexa Fluor 488\\
Donkey anti-rabbit IgG (H+L) & Invitrogen & 1:500 & Alexa Fluor 568\\
Donkey anti-mouse IgG (H+L) & Invitrogen & 1:500 & Alexa Fluor 568\\
\bottomrule
\end{tabular*}
\end{table}
Images in this thesis were acquired using Leica DMI6600 confocal laser scanning microscope with an HCX PL APO 63.0$\times$/1.40 oil objective and Leica Application Suite Advanced Fluorescence (v2.7.3 build 9725) software. As summarized in Table~\ref{tab:laser}, fluorophores were excited using 405 nm laser diode, Argon 488 nm laser (20\% power) or DPSS 561 nm laser and detected using photomultiplier tubes (PMT). The pinhole diameter was set to 95.6 {}\textmu m, scanning mode was unidirectional, line average was 2, sampling speed was 400 Hz. For co-localization studies the zoom was set to 5 and the voxel size was 48.1 nm (width) $\times$ 48.1 nm (height) $\times$ 125.9 nm (depth).

\begin{table}[h]
%\captionsetup{font=normalsize}
\caption{Settings for fluorescence excitation and detection.}
\label{tab:laser}
\small
\centering
\begin{tabular*}{\textwidth}{l@{\extracolsep{\fill}}lllll}
\toprule
\tabhead{Fluorophore} & \tabhead{Laser line} & \tabhead{Laser intensity} & \tabhead{PMT} & \tabhead{PMT gain} & \tabhead{PMT offset}\\
\midrule
DAPI & 405 nm & 8.00\% & 413-477 nm & 619 V & -1\\
Alexa 488 & 488 nm & 12.00\% & 506-598 nm & 646 V & -1\\
Alexa 568 & 561 nm & 15.00\% & 580-710 nm & 619 V & -1\\
\bottomrule
\end{tabular*}
\end{table}
The z-stack confocal microscopy images were further analyzed using ImageJ v1.51j. The stack viewing and color options were set as the default configuration of the Bio-Formats plugin. One z slice was selected in each image to represent all stacks. The brightness and contrast of the channels were uniformly adjusted in each staining experiment and a 10 {}\textmu m scale bar was added to the images.

% colocalization analysis: Imaris.
%----------------------------------------------------------------------------------------

% Define some commands to keep the formatting separated from the content 
 
% Chapter 3

\chapter{Results} % Main chapter title
\label{Chapter3} % For referencing the chapter elsewhere, use \ref{Chapter1} 

%----------------------------------------------------------------------------------------
% Define some commands to keep the formatting separated from the content 

% Chapter 4

\chapter{Discussion} % Main chapter title
\label{Chapter4} % For referencing the chapter elsewhere, use \ref{Chapter4} 

%When culturing cells in medium containing fetal bovine serum (FBS), please note that many lots of FBS contain tetracycline as FBS is generally isolated from cows that have been fed a diet containing tetracycline. If you culture your cells in medium containing FBS that is not reduced in tetracycline, you may observe low basal expression of your gene of interest in the absence of tetracycline

%Tetracycline (MW = 444.4) is commonly used as a broad spectrum antibiotic and acts to inhibit translation by blocking polypeptide chain elongation in bacteria. In the T-REx? System, tetracycline is used as an inducing agent to induce transcription of the gene of interest from the inducible expression vector. Tetracycline induces transcription by binding to the Tet repressor homodimer and causing the repressor to undergo a conformational change that renders it unable to bind to the Tet operator. The association constant of tetracycline to the Tet repressor is 3 x 109 M-1 (Takahashi et al., 1991). 

%Tetracycline is light sensitive

%You may want to vary the concentration of tetracycline (0.1 to 1 �g/ml) and time of exposure to tetracycline (8 to 24 hours) to optimize or modulate expression for your cell line.

%Doxycycline has been shown to have a longer half-life than tetracycline (48 hours vs. 24 hours, respectively). 
%----------------------------------------------------------------------------------------
% Define some commands to keep the formatting separated from the content 
 
%\include{Chapters/Chapter5} 

%----------------------------------------------------------------------------------------
%	ABBREVIATIONS
%----------------------------------------------------------------------------------------

\begin{abbreviations}{ll} % Include a list of abbreviations (a table of two columns)
\addchaptertocentry{\abbrevname}
\textbf{ABC} & Ammonium bicarbonate\\
\textbf{AGC} & Automatic gain control\\
\textbf{BAF} & Bafilomycin A1\\
\textbf{BioID} & Proximity-dependent biotin identification\\
\textbf{BSA} & Bovine serum albumin\\
\textbf{CCV} & Clathrin-coated visicle\\
\textbf{CPZ} & Chlorpromazine\\
\textbf{CSF} & Cerebrospinal fluid\\
\textbf{DAPI} & 4',6-Diamidino-2-phenylindole\\
\textbf{DMEM} & Dulbecco's modified Eagle's medium\\
\textbf{DOX} & Doxycycline\\
\textbf{DPSS} & Diode-pumped solid-state\\
\textbf{DTT} & Dithiothreitol\\
\textbf{EDTA} & Ethylenediaminetetraacetic acid\\
\textbf{EEA1} & Early endosome antigen 1\\
\textbf{GFP} & Green fluorescent protein\\
\textbf{GLUT1} & Facilitated glucose transporter member 1\\
\textbf{GO} & Gene ontology\\
\textbf{G1DS} & Glucose transporter 1 deficiency syndrome\\
\textbf{H} & Heavy\\
\textbf{HCMV} & Human cytomegalovirus\\
\textbf{HEK293} & Human embryonic kidney 293\\
\textbf{HEPES} & 4-(2-Hydroxyethyl)-1-piperazineethanesulfonic acid)\\
\textbf{HPLC} & High-pressure liquid chromatography\\
\textbf{hr} & Hour\\
\textbf{HRP} & Horseradish peroxidase\\
\textbf{IAA} & Iodoacemtamide\\
\textbf{L} & Light\\
\textbf{IgG} & Immunoglobulin G\\
\textbf{LAMP1} & Lysosomal-associated membrane protein 1\\
\textbf{LC3} & Microtubule-associated protein 1A/1B-light chain 3\\
\textbf{M} & Medium-heavy\\
\textbf{MS} & Mass spectrometry\\
\textbf{MS/MS} & Tandem mass spectrometry\\
\textbf{PBS} & Phosphate-buffered saline\\
\textbf{PCR} & Polymerase cain reaction\\
\textbf{PFA} & Paraformaldehyde\\
\textbf{PMT} & Photomultiplier tube\\
\textbf{PVDF} & Polyvinylidene fluoride\\
\textbf{rcf} & Relative centrifugal force\\
\textbf{RIPA} & Radioimmunoprecipitation assay\\
\textbf{rpm} & Revolutions per minute\\
\textbf{SDS-PAGE} & Sodium dodecyl sulfate polyacrylamide gel electrophoresis\\
\textbf{SILAC} & Stable isotope labeling by amino acids in cell culture\\
\textbf{SLC} & Solute carrier\\
\textbf{SNX} & Sorting nexin\\
\textbf{STX6} & Syntaxin 6\\
\textbf{TAP} & Tandem affinity purification\\
\textbf{TBST} & Tris buffered saline with Tween 20\\
\textbf{Tet} & Tetracycline\\
\textbf{TetR} & Tetracycline repressor\\
\textbf{Tf} & Transferrin\\
\textbf{TFA} & Trifluoroacetic acid\\
\textbf{TGN} & Trans-Golgi network\\
\textbf{TRFC} & Transferrin receptor\\
\textbf{tris} & Tris(hydroxymethyl)aminomethane\\
\textbf{WT} & Wild-type\\

\end{abbreviations}

%----------------------------------------------------------------------------------------
%	ACKNOWLEDGEMENTS
%----------------------------------------------------------------------------------------

\begin{acknowledgements}
\addchaptertocentry{\acknowledgementname} % Add the acknowledgements to the table of contents
This study was carried out in the Protein Dynamics Laboratory at Max Delbr\"{u}ck C	enter of Molecular Medicine. I would like to thank Professor Matthias Selbach for having given me the opportunity to complete my Master's thesis in his laboratory. I am also very grateful to my thesis supervisor Katrina Meyer who introduced me to this extraordinarily interesting project and offered me continuous support throughout the thesis. Her guidance and encouragement has been most inspiring and invaluable.

I would also like to thank Ouidad Benlasfer from the the laboratory of Professor Markus Landthaler for kindly providing the HEK293 stable cell lines. Additionally, I want to thank Giulia Russo from the department of Professor Volker Haucke for her help with immunofluorescence assays. For their great work and technical support I am very thankful to the Advanced Light Microscopy technology platform at MDC. Special thanks to Konstantin Grohmann for helping me with the colocalization analyses.

I would very much like to thank all the members of the lab. I have always felt welcome to ask for any help and support that I needed. I would like to express my special regards to Jo\~{a}o Fernandes, Dr.\ Michal Nadler-Holly, Dr.\ Matthias Ziehm and Martha Hergeselle for the inspiring and supportive discussions in the SBTH office. Alongside, I also thank Dr.\ Koshi Imami, Christian Sommer and the rest of the group members for providing a most pleasant working environment.

I want to thank my friends from the MolMed program and Tsinghua University for all the necessary distractions and laughters. I must thank our three cats, Salph, Twrx and Princess, who have always cheered me up with magic when I had difficulties. Finally, my deepest gratitude goes to my boyfriend Nimo, his and my family for understanding and encouraging me throughout the time it took to finish the present work.

\end{acknowledgements}

%----------------------------------------------------------------------------------------
%	THESIS CONTENT - APPENDICES
%----------------------------------------------------------------------------------------

%\appendix % Cue to tell LaTeX that the following "chapters" are Appendices

% Include the appendices of the thesis as separate files from the Appendices folder
% Uncomment the lines as you write the Appendices

%\include{Appendices/AppendixA}
%\include{Appendices/AppendixB}
%\include{Appendices/AppendixC}

%----------------------------------------------------------------------------------------
%	BIBLIOGRAPHY
%----------------------------------------------------------------------------------------
\begin{Literature}
\addchaptertocentry{\literaturename} 
\nocite{apsrev41Control}
\bibliography{main,revtex-custom}
\end{Literature}

%----------------------------------------------------------------------------------------
\begin{statement}
\addchaptertocentry{\statementname}
\vspace{10pt}
Herewith I confirm that I wrote this Master's Thesis in its entirety and that no additional assistance was provided, other than from the sources listed.
\end{statement}

\end{document}  
